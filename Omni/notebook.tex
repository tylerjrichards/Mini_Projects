
% Default to the notebook output style

    


% Inherit from the specified cell style.




    
\documentclass[11pt]{article}

    
    
    \usepackage[T1]{fontenc}
    % Nicer default font (+ math font) than Computer Modern for most use cases
    \usepackage{mathpazo}

    % Basic figure setup, for now with no caption control since it's done
    % automatically by Pandoc (which extracts ![](path) syntax from Markdown).
    \usepackage{graphicx}
    % We will generate all images so they have a width \maxwidth. This means
    % that they will get their normal width if they fit onto the page, but
    % are scaled down if they would overflow the margins.
    \makeatletter
    \def\maxwidth{\ifdim\Gin@nat@width>\linewidth\linewidth
    \else\Gin@nat@width\fi}
    \makeatother
    \let\Oldincludegraphics\includegraphics
    % Set max figure width to be 80% of text width, for now hardcoded.
    \renewcommand{\includegraphics}[1]{\Oldincludegraphics[width=.8\maxwidth]{#1}}
    % Ensure that by default, figures have no caption (until we provide a
    % proper Figure object with a Caption API and a way to capture that
    % in the conversion process - todo).
    \usepackage{caption}
    \DeclareCaptionLabelFormat{nolabel}{}
    \captionsetup{labelformat=nolabel}

    \usepackage{adjustbox} % Used to constrain images to a maximum size 
    \usepackage{xcolor} % Allow colors to be defined
    \usepackage{enumerate} % Needed for markdown enumerations to work
    \usepackage{geometry} % Used to adjust the document margins
    \usepackage{amsmath} % Equations
    \usepackage{amssymb} % Equations
    \usepackage{textcomp} % defines textquotesingle
    % Hack from http://tex.stackexchange.com/a/47451/13684:
    \AtBeginDocument{%
        \def\PYZsq{\textquotesingle}% Upright quotes in Pygmentized code
    }
    \usepackage{upquote} % Upright quotes for verbatim code
    \usepackage{eurosym} % defines \euro
    \usepackage[mathletters]{ucs} % Extended unicode (utf-8) support
    \usepackage[utf8x]{inputenc} % Allow utf-8 characters in the tex document
    \usepackage{fancyvrb} % verbatim replacement that allows latex
    \usepackage{grffile} % extends the file name processing of package graphics 
                         % to support a larger range 
    % The hyperref package gives us a pdf with properly built
    % internal navigation ('pdf bookmarks' for the table of contents,
    % internal cross-reference links, web links for URLs, etc.)
    \usepackage{hyperref}
    \usepackage{longtable} % longtable support required by pandoc >1.10
    \usepackage{booktabs}  % table support for pandoc > 1.12.2
    \usepackage[inline]{enumitem} % IRkernel/repr support (it uses the enumerate* environment)
    \usepackage[normalem]{ulem} % ulem is needed to support strikethroughs (\sout)
                                % normalem makes italics be italics, not underlines
    

    
    
    % Colors for the hyperref package
    \definecolor{urlcolor}{rgb}{0,.145,.698}
    \definecolor{linkcolor}{rgb}{.71,0.21,0.01}
    \definecolor{citecolor}{rgb}{.12,.54,.11}

    % ANSI colors
    \definecolor{ansi-black}{HTML}{3E424D}
    \definecolor{ansi-black-intense}{HTML}{282C36}
    \definecolor{ansi-red}{HTML}{E75C58}
    \definecolor{ansi-red-intense}{HTML}{B22B31}
    \definecolor{ansi-green}{HTML}{00A250}
    \definecolor{ansi-green-intense}{HTML}{007427}
    \definecolor{ansi-yellow}{HTML}{DDB62B}
    \definecolor{ansi-yellow-intense}{HTML}{B27D12}
    \definecolor{ansi-blue}{HTML}{208FFB}
    \definecolor{ansi-blue-intense}{HTML}{0065CA}
    \definecolor{ansi-magenta}{HTML}{D160C4}
    \definecolor{ansi-magenta-intense}{HTML}{A03196}
    \definecolor{ansi-cyan}{HTML}{60C6C8}
    \definecolor{ansi-cyan-intense}{HTML}{258F8F}
    \definecolor{ansi-white}{HTML}{C5C1B4}
    \definecolor{ansi-white-intense}{HTML}{A1A6B2}

    % commands and environments needed by pandoc snippets
    % extracted from the output of `pandoc -s`
    \providecommand{\tightlist}{%
      \setlength{\itemsep}{0pt}\setlength{\parskip}{0pt}}
    \DefineVerbatimEnvironment{Highlighting}{Verbatim}{commandchars=\\\{\}}
    % Add ',fontsize=\small' for more characters per line
    \newenvironment{Shaded}{}{}
    \newcommand{\KeywordTok}[1]{\textcolor[rgb]{0.00,0.44,0.13}{\textbf{{#1}}}}
    \newcommand{\DataTypeTok}[1]{\textcolor[rgb]{0.56,0.13,0.00}{{#1}}}
    \newcommand{\DecValTok}[1]{\textcolor[rgb]{0.25,0.63,0.44}{{#1}}}
    \newcommand{\BaseNTok}[1]{\textcolor[rgb]{0.25,0.63,0.44}{{#1}}}
    \newcommand{\FloatTok}[1]{\textcolor[rgb]{0.25,0.63,0.44}{{#1}}}
    \newcommand{\CharTok}[1]{\textcolor[rgb]{0.25,0.44,0.63}{{#1}}}
    \newcommand{\StringTok}[1]{\textcolor[rgb]{0.25,0.44,0.63}{{#1}}}
    \newcommand{\CommentTok}[1]{\textcolor[rgb]{0.38,0.63,0.69}{\textit{{#1}}}}
    \newcommand{\OtherTok}[1]{\textcolor[rgb]{0.00,0.44,0.13}{{#1}}}
    \newcommand{\AlertTok}[1]{\textcolor[rgb]{1.00,0.00,0.00}{\textbf{{#1}}}}
    \newcommand{\FunctionTok}[1]{\textcolor[rgb]{0.02,0.16,0.49}{{#1}}}
    \newcommand{\RegionMarkerTok}[1]{{#1}}
    \newcommand{\ErrorTok}[1]{\textcolor[rgb]{1.00,0.00,0.00}{\textbf{{#1}}}}
    \newcommand{\NormalTok}[1]{{#1}}
    
    % Additional commands for more recent versions of Pandoc
    \newcommand{\ConstantTok}[1]{\textcolor[rgb]{0.53,0.00,0.00}{{#1}}}
    \newcommand{\SpecialCharTok}[1]{\textcolor[rgb]{0.25,0.44,0.63}{{#1}}}
    \newcommand{\VerbatimStringTok}[1]{\textcolor[rgb]{0.25,0.44,0.63}{{#1}}}
    \newcommand{\SpecialStringTok}[1]{\textcolor[rgb]{0.73,0.40,0.53}{{#1}}}
    \newcommand{\ImportTok}[1]{{#1}}
    \newcommand{\DocumentationTok}[1]{\textcolor[rgb]{0.73,0.13,0.13}{\textit{{#1}}}}
    \newcommand{\AnnotationTok}[1]{\textcolor[rgb]{0.38,0.63,0.69}{\textbf{\textit{{#1}}}}}
    \newcommand{\CommentVarTok}[1]{\textcolor[rgb]{0.38,0.63,0.69}{\textbf{\textit{{#1}}}}}
    \newcommand{\VariableTok}[1]{\textcolor[rgb]{0.10,0.09,0.49}{{#1}}}
    \newcommand{\ControlFlowTok}[1]{\textcolor[rgb]{0.00,0.44,0.13}{\textbf{{#1}}}}
    \newcommand{\OperatorTok}[1]{\textcolor[rgb]{0.40,0.40,0.40}{{#1}}}
    \newcommand{\BuiltInTok}[1]{{#1}}
    \newcommand{\ExtensionTok}[1]{{#1}}
    \newcommand{\PreprocessorTok}[1]{\textcolor[rgb]{0.74,0.48,0.00}{{#1}}}
    \newcommand{\AttributeTok}[1]{\textcolor[rgb]{0.49,0.56,0.16}{{#1}}}
    \newcommand{\InformationTok}[1]{\textcolor[rgb]{0.38,0.63,0.69}{\textbf{\textit{{#1}}}}}
    \newcommand{\WarningTok}[1]{\textcolor[rgb]{0.38,0.63,0.69}{\textbf{\textit{{#1}}}}}
    
    
    % Define a nice break command that doesn't care if a line doesn't already
    % exist.
    \def\br{\hspace*{\fill} \\* }
    % Math Jax compatability definitions
    \def\gt{>}
    \def\lt{<}
    % Document parameters
    \title{Omni Recommendation System}
    
    
    

    % Pygments definitions
    
\makeatletter
\def\PY@reset{\let\PY@it=\relax \let\PY@bf=\relax%
    \let\PY@ul=\relax \let\PY@tc=\relax%
    \let\PY@bc=\relax \let\PY@ff=\relax}
\def\PY@tok#1{\csname PY@tok@#1\endcsname}
\def\PY@toks#1+{\ifx\relax#1\empty\else%
    \PY@tok{#1}\expandafter\PY@toks\fi}
\def\PY@do#1{\PY@bc{\PY@tc{\PY@ul{%
    \PY@it{\PY@bf{\PY@ff{#1}}}}}}}
\def\PY#1#2{\PY@reset\PY@toks#1+\relax+\PY@do{#2}}

\expandafter\def\csname PY@tok@w\endcsname{\def\PY@tc##1{\textcolor[rgb]{0.73,0.73,0.73}{##1}}}
\expandafter\def\csname PY@tok@c\endcsname{\let\PY@it=\textit\def\PY@tc##1{\textcolor[rgb]{0.25,0.50,0.50}{##1}}}
\expandafter\def\csname PY@tok@cp\endcsname{\def\PY@tc##1{\textcolor[rgb]{0.74,0.48,0.00}{##1}}}
\expandafter\def\csname PY@tok@k\endcsname{\let\PY@bf=\textbf\def\PY@tc##1{\textcolor[rgb]{0.00,0.50,0.00}{##1}}}
\expandafter\def\csname PY@tok@kp\endcsname{\def\PY@tc##1{\textcolor[rgb]{0.00,0.50,0.00}{##1}}}
\expandafter\def\csname PY@tok@kt\endcsname{\def\PY@tc##1{\textcolor[rgb]{0.69,0.00,0.25}{##1}}}
\expandafter\def\csname PY@tok@o\endcsname{\def\PY@tc##1{\textcolor[rgb]{0.40,0.40,0.40}{##1}}}
\expandafter\def\csname PY@tok@ow\endcsname{\let\PY@bf=\textbf\def\PY@tc##1{\textcolor[rgb]{0.67,0.13,1.00}{##1}}}
\expandafter\def\csname PY@tok@nb\endcsname{\def\PY@tc##1{\textcolor[rgb]{0.00,0.50,0.00}{##1}}}
\expandafter\def\csname PY@tok@nf\endcsname{\def\PY@tc##1{\textcolor[rgb]{0.00,0.00,1.00}{##1}}}
\expandafter\def\csname PY@tok@nc\endcsname{\let\PY@bf=\textbf\def\PY@tc##1{\textcolor[rgb]{0.00,0.00,1.00}{##1}}}
\expandafter\def\csname PY@tok@nn\endcsname{\let\PY@bf=\textbf\def\PY@tc##1{\textcolor[rgb]{0.00,0.00,1.00}{##1}}}
\expandafter\def\csname PY@tok@ne\endcsname{\let\PY@bf=\textbf\def\PY@tc##1{\textcolor[rgb]{0.82,0.25,0.23}{##1}}}
\expandafter\def\csname PY@tok@nv\endcsname{\def\PY@tc##1{\textcolor[rgb]{0.10,0.09,0.49}{##1}}}
\expandafter\def\csname PY@tok@no\endcsname{\def\PY@tc##1{\textcolor[rgb]{0.53,0.00,0.00}{##1}}}
\expandafter\def\csname PY@tok@nl\endcsname{\def\PY@tc##1{\textcolor[rgb]{0.63,0.63,0.00}{##1}}}
\expandafter\def\csname PY@tok@ni\endcsname{\let\PY@bf=\textbf\def\PY@tc##1{\textcolor[rgb]{0.60,0.60,0.60}{##1}}}
\expandafter\def\csname PY@tok@na\endcsname{\def\PY@tc##1{\textcolor[rgb]{0.49,0.56,0.16}{##1}}}
\expandafter\def\csname PY@tok@nt\endcsname{\let\PY@bf=\textbf\def\PY@tc##1{\textcolor[rgb]{0.00,0.50,0.00}{##1}}}
\expandafter\def\csname PY@tok@nd\endcsname{\def\PY@tc##1{\textcolor[rgb]{0.67,0.13,1.00}{##1}}}
\expandafter\def\csname PY@tok@s\endcsname{\def\PY@tc##1{\textcolor[rgb]{0.73,0.13,0.13}{##1}}}
\expandafter\def\csname PY@tok@sd\endcsname{\let\PY@it=\textit\def\PY@tc##1{\textcolor[rgb]{0.73,0.13,0.13}{##1}}}
\expandafter\def\csname PY@tok@si\endcsname{\let\PY@bf=\textbf\def\PY@tc##1{\textcolor[rgb]{0.73,0.40,0.53}{##1}}}
\expandafter\def\csname PY@tok@se\endcsname{\let\PY@bf=\textbf\def\PY@tc##1{\textcolor[rgb]{0.73,0.40,0.13}{##1}}}
\expandafter\def\csname PY@tok@sr\endcsname{\def\PY@tc##1{\textcolor[rgb]{0.73,0.40,0.53}{##1}}}
\expandafter\def\csname PY@tok@ss\endcsname{\def\PY@tc##1{\textcolor[rgb]{0.10,0.09,0.49}{##1}}}
\expandafter\def\csname PY@tok@sx\endcsname{\def\PY@tc##1{\textcolor[rgb]{0.00,0.50,0.00}{##1}}}
\expandafter\def\csname PY@tok@m\endcsname{\def\PY@tc##1{\textcolor[rgb]{0.40,0.40,0.40}{##1}}}
\expandafter\def\csname PY@tok@gh\endcsname{\let\PY@bf=\textbf\def\PY@tc##1{\textcolor[rgb]{0.00,0.00,0.50}{##1}}}
\expandafter\def\csname PY@tok@gu\endcsname{\let\PY@bf=\textbf\def\PY@tc##1{\textcolor[rgb]{0.50,0.00,0.50}{##1}}}
\expandafter\def\csname PY@tok@gd\endcsname{\def\PY@tc##1{\textcolor[rgb]{0.63,0.00,0.00}{##1}}}
\expandafter\def\csname PY@tok@gi\endcsname{\def\PY@tc##1{\textcolor[rgb]{0.00,0.63,0.00}{##1}}}
\expandafter\def\csname PY@tok@gr\endcsname{\def\PY@tc##1{\textcolor[rgb]{1.00,0.00,0.00}{##1}}}
\expandafter\def\csname PY@tok@ge\endcsname{\let\PY@it=\textit}
\expandafter\def\csname PY@tok@gs\endcsname{\let\PY@bf=\textbf}
\expandafter\def\csname PY@tok@gp\endcsname{\let\PY@bf=\textbf\def\PY@tc##1{\textcolor[rgb]{0.00,0.00,0.50}{##1}}}
\expandafter\def\csname PY@tok@go\endcsname{\def\PY@tc##1{\textcolor[rgb]{0.53,0.53,0.53}{##1}}}
\expandafter\def\csname PY@tok@gt\endcsname{\def\PY@tc##1{\textcolor[rgb]{0.00,0.27,0.87}{##1}}}
\expandafter\def\csname PY@tok@err\endcsname{\def\PY@bc##1{\setlength{\fboxsep}{0pt}\fcolorbox[rgb]{1.00,0.00,0.00}{1,1,1}{\strut ##1}}}
\expandafter\def\csname PY@tok@kc\endcsname{\let\PY@bf=\textbf\def\PY@tc##1{\textcolor[rgb]{0.00,0.50,0.00}{##1}}}
\expandafter\def\csname PY@tok@kd\endcsname{\let\PY@bf=\textbf\def\PY@tc##1{\textcolor[rgb]{0.00,0.50,0.00}{##1}}}
\expandafter\def\csname PY@tok@kn\endcsname{\let\PY@bf=\textbf\def\PY@tc##1{\textcolor[rgb]{0.00,0.50,0.00}{##1}}}
\expandafter\def\csname PY@tok@kr\endcsname{\let\PY@bf=\textbf\def\PY@tc##1{\textcolor[rgb]{0.00,0.50,0.00}{##1}}}
\expandafter\def\csname PY@tok@bp\endcsname{\def\PY@tc##1{\textcolor[rgb]{0.00,0.50,0.00}{##1}}}
\expandafter\def\csname PY@tok@fm\endcsname{\def\PY@tc##1{\textcolor[rgb]{0.00,0.00,1.00}{##1}}}
\expandafter\def\csname PY@tok@vc\endcsname{\def\PY@tc##1{\textcolor[rgb]{0.10,0.09,0.49}{##1}}}
\expandafter\def\csname PY@tok@vg\endcsname{\def\PY@tc##1{\textcolor[rgb]{0.10,0.09,0.49}{##1}}}
\expandafter\def\csname PY@tok@vi\endcsname{\def\PY@tc##1{\textcolor[rgb]{0.10,0.09,0.49}{##1}}}
\expandafter\def\csname PY@tok@vm\endcsname{\def\PY@tc##1{\textcolor[rgb]{0.10,0.09,0.49}{##1}}}
\expandafter\def\csname PY@tok@sa\endcsname{\def\PY@tc##1{\textcolor[rgb]{0.73,0.13,0.13}{##1}}}
\expandafter\def\csname PY@tok@sb\endcsname{\def\PY@tc##1{\textcolor[rgb]{0.73,0.13,0.13}{##1}}}
\expandafter\def\csname PY@tok@sc\endcsname{\def\PY@tc##1{\textcolor[rgb]{0.73,0.13,0.13}{##1}}}
\expandafter\def\csname PY@tok@dl\endcsname{\def\PY@tc##1{\textcolor[rgb]{0.73,0.13,0.13}{##1}}}
\expandafter\def\csname PY@tok@s2\endcsname{\def\PY@tc##1{\textcolor[rgb]{0.73,0.13,0.13}{##1}}}
\expandafter\def\csname PY@tok@sh\endcsname{\def\PY@tc##1{\textcolor[rgb]{0.73,0.13,0.13}{##1}}}
\expandafter\def\csname PY@tok@s1\endcsname{\def\PY@tc##1{\textcolor[rgb]{0.73,0.13,0.13}{##1}}}
\expandafter\def\csname PY@tok@mb\endcsname{\def\PY@tc##1{\textcolor[rgb]{0.40,0.40,0.40}{##1}}}
\expandafter\def\csname PY@tok@mf\endcsname{\def\PY@tc##1{\textcolor[rgb]{0.40,0.40,0.40}{##1}}}
\expandafter\def\csname PY@tok@mh\endcsname{\def\PY@tc##1{\textcolor[rgb]{0.40,0.40,0.40}{##1}}}
\expandafter\def\csname PY@tok@mi\endcsname{\def\PY@tc##1{\textcolor[rgb]{0.40,0.40,0.40}{##1}}}
\expandafter\def\csname PY@tok@il\endcsname{\def\PY@tc##1{\textcolor[rgb]{0.40,0.40,0.40}{##1}}}
\expandafter\def\csname PY@tok@mo\endcsname{\def\PY@tc##1{\textcolor[rgb]{0.40,0.40,0.40}{##1}}}
\expandafter\def\csname PY@tok@ch\endcsname{\let\PY@it=\textit\def\PY@tc##1{\textcolor[rgb]{0.25,0.50,0.50}{##1}}}
\expandafter\def\csname PY@tok@cm\endcsname{\let\PY@it=\textit\def\PY@tc##1{\textcolor[rgb]{0.25,0.50,0.50}{##1}}}
\expandafter\def\csname PY@tok@cpf\endcsname{\let\PY@it=\textit\def\PY@tc##1{\textcolor[rgb]{0.25,0.50,0.50}{##1}}}
\expandafter\def\csname PY@tok@c1\endcsname{\let\PY@it=\textit\def\PY@tc##1{\textcolor[rgb]{0.25,0.50,0.50}{##1}}}
\expandafter\def\csname PY@tok@cs\endcsname{\let\PY@it=\textit\def\PY@tc##1{\textcolor[rgb]{0.25,0.50,0.50}{##1}}}

\def\PYZbs{\char`\\}
\def\PYZus{\char`\_}
\def\PYZob{\char`\{}
\def\PYZcb{\char`\}}
\def\PYZca{\char`\^}
\def\PYZam{\char`\&}
\def\PYZlt{\char`\<}
\def\PYZgt{\char`\>}
\def\PYZsh{\char`\#}
\def\PYZpc{\char`\%}
\def\PYZdl{\char`\$}
\def\PYZhy{\char`\-}
\def\PYZsq{\char`\'}
\def\PYZdq{\char`\"}
\def\PYZti{\char`\~}
% for compatibility with earlier versions
\def\PYZat{@}
\def\PYZlb{[}
\def\PYZrb{]}
\makeatother


    % Exact colors from NB
    \definecolor{incolor}{rgb}{0.0, 0.0, 0.5}
    \definecolor{outcolor}{rgb}{0.545, 0.0, 0.0}



    
    % Prevent overflowing lines due to hard-to-break entities
    \sloppy 
    % Setup hyperref package
    \hypersetup{
      breaklinks=true,  % so long urls are correctly broken across lines
      colorlinks=true,
      urlcolor=urlcolor,
      linkcolor=linkcolor,
      citecolor=citecolor,
      }
    % Slightly bigger margins than the latex defaults
    
    \geometry{verbose,tmargin=1in,bmargin=1in,lmargin=1in,rmargin=1in}
    
    

    \begin{document}
    
    
    \maketitle
    
    

    
    \hypertarget{recommendation-systems-at-omni}{%
\section{Recommendation Systems at
Omni}\label{recommendation-systems-at-omni}}

\hypertarget{by-tyler-richards}{%
\subsubsection{By Tyler Richards}\label{by-tyler-richards}}

    About two weeks ago, I was asked how I think about recommendation
systems by Thomas and I wanted to make a more established effort at it.
So I built two recommendation systems using a small bit of data I made
up answering two recommendation problems that I think Omni has.

\begin{enumerate}
\def\labelenumi{\arabic{enumi}.}
\tightlist
\item
  A user returns to the home page, what products should be recommended?
\item
  A user adds an item to their cart or rents a product, what next item
  should be recommended?
\end{enumerate}

Please note: I have very little data on Omni and only know that they
have a table of users (IDs, email, sign up date, etc) along with a table
of events (rentals, views, etc).

    \hypertarget{user-item-recommendation-algorithm}{%
\subsection{User-item recommendation
algorithm}\label{user-item-recommendation-algorithm}}

    The fundamental question here is ``What is the most similar user and
item combination to the set of data Omni already has?'' Let's explore
what is called a collaborative filtering mechanism based on some fake
data I've come up with.

I'll print it out below!

    \begin{Verbatim}[commandchars=\\\{\}]
{\color{incolor}In [{\color{incolor}89}]:} \PY{c+c1}{\PYZsh{}this is a subset of data, where 0 means the item has not been rented and a 1 means the item has. }
         \PY{n}{User\PYZus{}ratings} \PY{o}{=} \PY{n}{pd}\PY{o}{.}\PY{n}{DataFrame}\PY{p}{(}\PY{n}{np}\PY{o}{.}\PY{n}{asarray}\PY{p}{(}\PY{p}{[}\PY{p}{[}\PY{l+m+mi}{1}\PY{p}{,}\PY{l+m+mi}{1}\PY{p}{,}\PY{l+m+mi}{1}\PY{p}{,}\PY{l+m+mi}{0}\PY{p}{,}\PY{l+m+mi}{1}\PY{p}{,}\PY{l+m+mi}{0}\PY{p}{]}\PY{p}{,} 
                         \PY{p}{[}\PY{l+m+mi}{1}\PY{p}{,}\PY{l+m+mi}{0}\PY{p}{,}\PY{l+m+mi}{1}\PY{p}{,}\PY{l+m+mi}{0}\PY{p}{,}\PY{l+m+mi}{0}\PY{p}{,}\PY{l+m+mi}{0}\PY{p}{]}\PY{p}{,}
                        \PY{p}{[}\PY{l+m+mi}{0}\PY{p}{,}\PY{l+m+mi}{1}\PY{p}{,}\PY{l+m+mi}{0}\PY{p}{,}\PY{l+m+mi}{0}\PY{p}{,}\PY{l+m+mi}{0}\PY{p}{,}\PY{l+m+mi}{0}\PY{p}{]}\PY{p}{,}
                        \PY{p}{[}\PY{l+m+mi}{0}\PY{p}{,}\PY{l+m+mi}{1}\PY{p}{,}\PY{l+m+mi}{0}\PY{p}{,}\PY{l+m+mi}{1}\PY{p}{,}\PY{l+m+mi}{0}\PY{p}{,}\PY{l+m+mi}{1}\PY{p}{]}\PY{p}{,}
                        \PY{p}{[}\PY{l+m+mi}{1}\PY{p}{,}\PY{l+m+mi}{0}\PY{p}{,}\PY{l+m+mi}{1}\PY{p}{,}\PY{l+m+mi}{0}\PY{p}{,}\PY{l+m+mi}{1}\PY{p}{,}\PY{l+m+mi}{0}\PY{p}{]}\PY{p}{,}
                        \PY{p}{[}\PY{l+m+mi}{1}\PY{p}{,}\PY{l+m+mi}{0}\PY{p}{,}\PY{l+m+mi}{0}\PY{p}{,}\PY{l+m+mi}{0}\PY{p}{,}\PY{l+m+mi}{0}\PY{p}{,}\PY{l+m+mi}{0}\PY{p}{]}\PY{p}{]}\PY{p}{)}\PY{p}{)}
         \PY{n}{User\PYZus{}ratings}\PY{o}{.}\PY{n}{columns} \PY{o}{=} \PY{p}{[}\PY{l+s+s2}{\PYZdq{}}\PY{l+s+s2}{Camping Stove}\PY{l+s+s2}{\PYZdq{}}\PY{p}{,} \PY{l+s+s2}{\PYZdq{}}\PY{l+s+s2}{Sleeping Bag}\PY{l+s+s2}{\PYZdq{}}\PY{p}{,} \PY{l+s+s2}{\PYZdq{}}\PY{l+s+s2}{Cooler}\PY{l+s+s2}{\PYZdq{}}\PY{p}{,} \PY{l+s+s2}{\PYZdq{}}\PY{l+s+s2}{Inner Tube}\PY{l+s+s2}{\PYZdq{}}\PY{p}{,} \PY{l+s+s2}{\PYZdq{}}\PY{l+s+s2}{Plastic Tub}\PY{l+s+s2}{\PYZdq{}}\PY{p}{,} \PY{l+s+s2}{\PYZdq{}}\PY{l+s+s2}{Camping Book}\PY{l+s+s2}{\PYZdq{}}\PY{p}{]}
         \PY{n}{User\PYZus{}ratings}
\end{Verbatim}


\begin{Verbatim}[commandchars=\\\{\}]
{\color{outcolor}Out[{\color{outcolor}89}]:}    Camping Stove  Sleeping Bag  Cooler  Inner Tube  Plastic Tub  Camping Book
         0              1             1       1           0            1             0
         1              1             0       1           0            0             0
         2              0             1       0           0            0             0
         3              0             1       0           1            0             1
         4              1             0       1           0            1             0
         5              1             0       0           0            0             0
\end{Verbatim}
            
    To pick the items, I just picked 6 of the top items on the Omni website
and used the numbers 0 through 5 as the User IDs.\\
Each value represents how many times a user rented the product, data
that I would get using a cross tab query depending on the database
structure.

    Now our question is, user 5 has rented the Camping Stove, what should we
recommend to them next? The way the chosen method works is we first take
the similarity between each item (using cosine pairwise distance), which
we will use for our k nearest neighbor algorithm a little later.

    \begin{Verbatim}[commandchars=\\\{\}]
{\color{incolor}In [{\color{incolor}90}]:} \PY{n}{cosine\PYZus{}sim} \PY{o}{=} \PY{n}{pd}\PY{o}{.}\PY{n}{DataFrame}\PY{p}{(}\PY{l+m+mi}{1}\PY{o}{\PYZhy{}}\PY{n}{pairwise\PYZus{}distances}\PY{p}{(}\PY{n}{User\PYZus{}ratings}\PY{p}{,} \PY{n}{metric}\PY{o}{=}\PY{l+s+s2}{\PYZdq{}}\PY{l+s+s2}{cosine}\PY{l+s+s2}{\PYZdq{}}\PY{p}{)}\PY{p}{)}
\end{Verbatim}


    \begin{Verbatim}[commandchars=\\\{\}]
{\color{incolor}In [{\color{incolor}91}]:} \PY{n}{cosine\PYZus{}sim}
\end{Verbatim}


\begin{Verbatim}[commandchars=\\\{\}]
{\color{outcolor}Out[{\color{outcolor}91}]:}           0         1        2         3         4         5
         0  1.000000  0.707107  0.50000  0.288675  0.866025  0.500000
         1  0.707107  1.000000  0.00000  0.000000  0.816497  0.707107
         2  0.500000  0.000000  1.00000  0.577350  0.000000  0.000000
         3  0.288675  0.000000  0.57735  1.000000  0.000000  0.000000
         4  0.866025  0.816497  0.00000  0.000000  1.000000  0.577350
         5  0.500000  0.707107  0.00000  0.000000  0.577350  1.000000
\end{Verbatim}
            
    Now that we know how ``far'' each item is from each other, we can
utilize a Nearest Neighbors algorithm to recommend a rating.\\
This is a two step process, first we need to get the closest user in
terms of similarity and then we can get the item based on that closest
user. The code for this is a couple dozen lines long as is fairly boring
and doesn't help the reader understand how I think about this
algorithmic approach for Omni any better, so I put it at the very bottom
if anyone wants to see the nuts and bolts. The image below shows a
snippet of how our nearest neighbors classification algorithm works!

    \begin{Verbatim}[commandchars=\\\{\}]
{\color{incolor}In [{\color{incolor}54}]:} \PY{n}{Image}\PY{p}{(}\PY{n}{url} \PY{o}{=} \PY{l+s+s1}{\PYZsq{}}\PY{l+s+s1}{https://i.pinimg.com/originals/65/36/b9/6536b9a63fc427e0fc3e1a9687b49aff.png}\PY{l+s+s1}{\PYZsq{}}\PY{p}{,} \PY{n}{height}\PY{o}{=}\PY{p}{)}
\end{Verbatim}


\begin{Verbatim}[commandchars=\\\{\}]
{\color{outcolor}Out[{\color{outcolor}54}]:} <IPython.core.display.Image object>
\end{Verbatim}
            
    Let's see a prediction for the 5th user

    \begin{Verbatim}[commandchars=\\\{\}]
{\color{incolor}In [{\color{incolor}98}]:} \PY{n}{predict\PYZus{}userbased}\PY{p}{(}\PY{l+m+mi}{5}\PY{p}{,}\PY{l+m+mi}{1}\PY{p}{,} \PY{n}{User\PYZus{}ratings}\PY{p}{)}
\end{Verbatim}


    \begin{Verbatim}[commandchars=\\\{\}]

Predicted rating for user 5 -> item 1: 1

    \end{Verbatim}

\begin{Verbatim}[commandchars=\\\{\}]
{\color{outcolor}Out[{\color{outcolor}98}]:} 1
\end{Verbatim}
            
    Here is the result of this algorithm, we predict that the 5th user, who
had rented a camping stove, would like to rent a Sleeping Bag. Ideally,
we would have way more data and exclude users who have only rented one
thing as the confidence in our prediction will be low in this case.
These sort of ratings might make sense as their own section on the home
page, it might look something like this.

    \begin{Verbatim}[commandchars=\\\{\}]
{\color{incolor}In [{\color{incolor}48}]:} \PY{n}{Image}\PY{p}{(}\PY{n}{filename}\PY{o}{=}\PY{l+s+s1}{\PYZsq{}}\PY{l+s+s1}{New\PYZus{}Page.png}\PY{l+s+s1}{\PYZsq{}}\PY{p}{)}
\end{Verbatim}

\texttt{\color{outcolor}Out[{\color{outcolor}48}]:}
    
    \begin{center}
    \adjustimage{max size={0.9\linewidth}{0.9\paperheight}}{output_14_0.png}
    \end{center}
    { \hspace*{\fill} \\}
    

    N.B. there are dozens of other considerations to make here, one of which
is time. Ideally, the user comparison recommendation algorithm will
encourage users to take new experiences and could be used to prompt
users to the site (via email notification, etc) but would generally fail
in the same way that the unsupervised learning algorithm (knn) might, it
would recommend items useful at the same time as previous rentals
(cooler combined with grill and sleeping bag) but might be hesitant to
recommend new areas to rent in. This all depends on user behavior, which
I would use to expand the scope of this method.

    \hypertarget{next-item-recommendation}{%
\subsection{Next Item Recommendation}\label{next-item-recommendation}}

    After exploring Omni's open rentals page, I found that Omni already
implemented a psuedo-recommendation engine by creating kits. The next
question is, can we use the item similarities we have already
established to automatically create these and can we recommend another
item to rent after the first item is rented to increase basket size?
Let's check it out.

    \begin{Verbatim}[commandchars=\\\{\}]
{\color{incolor}In [{\color{incolor}3}]:} \PY{n}{Image}\PY{p}{(}\PY{n}{filename}\PY{o}{=}\PY{l+s+s1}{\PYZsq{}}\PY{l+s+s1}{Combo.png}\PY{l+s+s1}{\PYZsq{}}\PY{p}{)}
\end{Verbatim}

\texttt{\color{outcolor}Out[{\color{outcolor}3}]:}
    
    \begin{center}
    \adjustimage{max size={0.9\linewidth}{0.9\paperheight}}{output_18_0.png}
    \end{center}
    { \hspace*{\fill} \\}
    

    We can also use collaborative filtering on a item to item basis to
predict the k most similar items to help make Kits like these, or we can
recommend items after a user rents an item. The implementation for this
is smaller than the implementation for the user based rec system, and is
directly below.

    \begin{Verbatim}[commandchars=\\\{\}]
{\color{incolor}In [{\color{incolor}99}]:} \PY{c+c1}{\PYZsh{}the actual function for the item prediction}
         \PY{k}{def} \PY{n+nf}{find\PYZus{}k\PYZus{}similar\PYZus{}items}\PY{p}{(}\PY{n}{item\PYZus{}id}\PY{p}{,} \PY{n}{ratings}\PY{p}{,} \PY{n}{metric}\PY{o}{=}\PY{l+s+s1}{\PYZsq{}}\PY{l+s+s1}{cosine}\PY{l+s+s1}{\PYZsq{}}\PY{p}{,} \PY{n}{k}\PY{o}{=}\PY{l+m+mi}{2}\PY{p}{)}\PY{p}{:}
             \PY{n}{similarities}\PY{o}{=}\PY{p}{[}\PY{p}{]}
             \PY{n}{indices}\PY{o}{=}\PY{p}{[}\PY{p}{]}    
             \PY{n}{ratings}\PY{o}{=}\PY{n}{ratings}\PY{o}{.}\PY{n}{T}
             \PY{n}{model\PYZus{}knn} \PY{o}{=} \PY{n}{NearestNeighbors}\PY{p}{(}\PY{n}{metric} \PY{o}{=} \PY{n}{metric}\PY{p}{,} \PY{n}{algorithm} \PY{o}{=} \PY{l+s+s1}{\PYZsq{}}\PY{l+s+s1}{brute}\PY{l+s+s1}{\PYZsq{}}\PY{p}{)}
             \PY{n}{model\PYZus{}knn}\PY{o}{.}\PY{n}{fit}\PY{p}{(}\PY{n}{ratings}\PY{p}{)}
             \PY{n}{distances}\PY{p}{,} \PY{n}{indices} \PY{o}{=} \PY{n}{model\PYZus{}knn}\PY{o}{.}\PY{n}{kneighbors}\PY{p}{(}\PY{n}{ratings}\PY{o}{.}\PY{n}{iloc}\PY{p}{[}\PY{n}{item\PYZus{}id}\PY{o}{\PYZhy{}}\PY{l+m+mi}{1}\PY{p}{,} \PY{p}{:}\PY{p}{]}\PY{o}{.}\PY{n}{values}\PY{o}{.}\PY{n}{reshape}\PY{p}{(}\PY{l+m+mi}{1}\PY{p}{,} \PY{o}{\PYZhy{}}\PY{l+m+mi}{1}\PY{p}{)}\PY{p}{,} \PY{n}{n\PYZus{}neighbors} \PY{o}{=} \PY{n}{k}\PY{o}{+}\PY{l+m+mi}{1}\PY{p}{)}
             \PY{n}{similarities} \PY{o}{=} \PY{l+m+mi}{1}\PY{o}{\PYZhy{}}\PY{n}{distances}\PY{o}{.}\PY{n}{flatten}\PY{p}{(}\PY{p}{)}
             \PY{n+nb}{print}\PY{p}{(}\PY{l+s+s1}{\PYZsq{}}\PY{l+s+si}{\PYZob{}0\PYZcb{}}\PY{l+s+s1}{ most similar items for item }\PY{l+s+si}{\PYZob{}1\PYZcb{}}\PY{l+s+s1}{:}\PY{l+s+se}{\PYZbs{}n}\PY{l+s+s1}{\PYZsq{}}\PY{o}{.}\PY{n}{format}\PY{p}{(}\PY{n}{k}\PY{p}{,}\PY{n}{item\PYZus{}id}\PY{p}{)}\PY{p}{)}
             \PY{k}{for} \PY{n}{i} \PY{o+ow}{in} \PY{n+nb}{range}\PY{p}{(}\PY{l+m+mi}{0}\PY{p}{,} \PY{n+nb}{len}\PY{p}{(}\PY{n}{indices}\PY{o}{.}\PY{n}{flatten}\PY{p}{(}\PY{p}{)}\PY{p}{)}\PY{p}{)}\PY{p}{:}
                 \PY{k}{if} \PY{n}{indices}\PY{o}{.}\PY{n}{flatten}\PY{p}{(}\PY{p}{)}\PY{p}{[}\PY{n}{i}\PY{p}{]}\PY{o}{+}\PY{l+m+mi}{1} \PY{o}{==} \PY{n}{item\PYZus{}id}\PY{p}{:}
                     \PY{k}{continue}\PY{p}{;}
         
                 \PY{k}{else}\PY{p}{:}
                     \PY{n+nb}{print}\PY{p}{(}\PY{l+s+s1}{\PYZsq{}}\PY{l+s+si}{\PYZob{}0\PYZcb{}}\PY{l+s+s1}{: Item }\PY{l+s+si}{\PYZob{}1\PYZcb{}}\PY{l+s+s1}{ :, with similarity of }\PY{l+s+si}{\PYZob{}2\PYZcb{}}\PY{l+s+s1}{\PYZsq{}}\PY{o}{.}\PY{n}{format}\PY{p}{(}\PY{n}{i}\PY{p}{,}\PY{n}{indices}\PY{o}{.}\PY{n}{flatten}\PY{p}{(}\PY{p}{)}\PY{p}{[}\PY{n}{i}\PY{p}{]}\PY{o}{+}\PY{l+m+mi}{1}\PY{p}{,} \PY{n}{similarities}\PY{o}{.}\PY{n}{flatten}\PY{p}{(}\PY{p}{)}\PY{p}{[}\PY{n}{i}\PY{p}{]}\PY{p}{)}\PY{p}{)}
             \PY{k}{return} \PY{n}{similarities}\PY{p}{,}\PY{n}{indices}
\end{Verbatim}


    \begin{Verbatim}[commandchars=\\\{\}]
{\color{incolor}In [{\color{incolor}102}]:} \PY{n}{find\PYZus{}k\PYZus{}similar\PYZus{}items}\PY{p}{(}\PY{l+m+mi}{0}\PY{p}{,} \PY{n}{User\PYZus{}ratings}\PY{p}{)}
\end{Verbatim}


    \begin{Verbatim}[commandchars=\\\{\}]
2 most similar items for item 0:

0: Item 4 :, with similarity of 1.0
1: Item 6 :, with similarity of 1.0
2: Item 2 :, with similarity of 0.5773502691896258

    \end{Verbatim}

\begin{Verbatim}[commandchars=\\\{\}]
{\color{outcolor}Out[{\color{outcolor}102}]:} (array([1.        , 1.        , 0.57735027]), array([[3, 5, 1]]))
\end{Verbatim}
            
    So if we only recommend based on item similarities, we find that people
who rented stoves would also like Inner Tubes and Camping Books (this
doesn't make much sense, but it based on fake data).

    I would like to thank the reader for getting to this point, as I wanted
to show how I think about recommendation systems specific to Omni. As I
talked about before, I'm always free to talk about anything written here
and hope to talk to you all soon.

    \hypertarget{appendix-with-code-snippets-and-library-imports}{%
\subsection{Appendix with code snippets and library
imports}\label{appendix-with-code-snippets-and-library-imports}}

    \begin{Verbatim}[commandchars=\\\{\}]
{\color{incolor}In [{\color{incolor}93}]:} \PY{k}{def} \PY{n+nf}{find\PYZus{}k\PYZus{}similar\PYZus{}users}\PY{p}{(}\PY{n}{user\PYZus{}id}\PY{p}{,} \PY{n}{ratings}\PY{p}{,} \PY{n}{metric} \PY{o}{=} \PY{l+s+s1}{\PYZsq{}}\PY{l+s+s1}{cosine}\PY{l+s+s1}{\PYZsq{}}\PY{p}{,} \PY{n}{k}\PY{o}{=}\PY{l+m+mi}{1}\PY{p}{)}\PY{p}{:}
             \PY{n}{similarities}\PY{o}{=}\PY{p}{[}\PY{p}{]}
             \PY{n}{indices}\PY{o}{=}\PY{p}{[}\PY{p}{]}
             \PY{n}{model\PYZus{}knn} \PY{o}{=} \PY{n}{NearestNeighbors}\PY{p}{(}\PY{n}{metric} \PY{o}{=} \PY{n}{metric}\PY{p}{,} \PY{n}{algorithm} \PY{o}{=} \PY{l+s+s1}{\PYZsq{}}\PY{l+s+s1}{brute}\PY{l+s+s1}{\PYZsq{}}\PY{p}{)} 
             \PY{n}{model\PYZus{}knn}\PY{o}{.}\PY{n}{fit}\PY{p}{(}\PY{n}{ratings}\PY{p}{)}
         
             \PY{n}{distances}\PY{p}{,} \PY{n}{indices} \PY{o}{=} \PY{n}{model\PYZus{}knn}\PY{o}{.}\PY{n}{kneighbors}\PY{p}{(}\PY{n}{ratings}\PY{o}{.}\PY{n}{iloc}\PY{p}{[}\PY{n}{user\PYZus{}id}\PY{o}{\PYZhy{}}\PY{l+m+mi}{1}\PY{p}{,} \PY{p}{:}\PY{p}{]}\PY{o}{.}\PY{n}{values}\PY{o}{.}\PY{n}{reshape}\PY{p}{(}\PY{l+m+mi}{1}\PY{p}{,} \PY{o}{\PYZhy{}}\PY{l+m+mi}{1}\PY{p}{)}\PY{p}{,} \PY{n}{n\PYZus{}neighbors} \PY{o}{=} \PY{n}{k}\PY{o}{+}\PY{l+m+mi}{1}\PY{p}{)}
             \PY{n}{similarities} \PY{o}{=} \PY{l+m+mi}{1}\PY{o}{\PYZhy{}}\PY{n}{distances}\PY{o}{.}\PY{n}{flatten}\PY{p}{(}\PY{p}{)}
             \PY{k}{return} \PY{n}{similarities}\PY{p}{,}\PY{n}{indices}
\end{Verbatim}


    \begin{Verbatim}[commandchars=\\\{\}]
{\color{incolor}In [{\color{incolor}97}]:} \PY{n}{find\PYZus{}k\PYZus{}similar\PYZus{}users}\PY{p}{(}\PY{l+m+mi}{5}\PY{p}{,} \PY{n}{User\PYZus{}ratings}\PY{p}{)}
\end{Verbatim}


\begin{Verbatim}[commandchars=\\\{\}]
{\color{outcolor}Out[{\color{outcolor}97}]:} (array([1.       , 0.8660254]), array([[4, 0]]))
\end{Verbatim}
            
    \begin{Verbatim}[commandchars=\\\{\}]
{\color{incolor}In [{\color{incolor}94}]:} \PY{k}{def} \PY{n+nf}{predict\PYZus{}userbased}\PY{p}{(}\PY{n}{user\PYZus{}id}\PY{p}{,} \PY{n}{item\PYZus{}id}\PY{p}{,} \PY{n}{ratings}\PY{p}{,} \PY{n}{metric} \PY{o}{=} \PY{l+s+s1}{\PYZsq{}}\PY{l+s+s1}{cosine}\PY{l+s+s1}{\PYZsq{}}\PY{p}{,} \PY{n}{k}\PY{o}{=}\PY{l+m+mi}{1}\PY{p}{)}\PY{p}{:}
             \PY{n}{prediction}\PY{o}{=}\PY{l+m+mi}{0}
             \PY{n}{similarities}\PY{p}{,} \PY{n}{indices}\PY{o}{=}\PY{n}{find\PYZus{}k\PYZus{}similar\PYZus{}users}\PY{p}{(}\PY{n}{user\PYZus{}id}\PY{p}{,} \PY{n}{ratings}\PY{p}{,}\PY{n}{metric}\PY{p}{,} \PY{n}{k}\PY{p}{)} \PY{c+c1}{\PYZsh{}similar users based on cosine similarity}
             \PY{n}{mean\PYZus{}rating} \PY{o}{=} \PY{n}{ratings}\PY{o}{.}\PY{n}{loc}\PY{p}{[}\PY{n}{user\PYZus{}id}\PY{o}{\PYZhy{}}\PY{l+m+mi}{1}\PY{p}{,}\PY{p}{:}\PY{p}{]}\PY{o}{.}\PY{n}{mean}\PY{p}{(}\PY{p}{)} \PY{c+c1}{\PYZsh{}to adjust for zero based indexing}
             \PY{n}{sum\PYZus{}wt} \PY{o}{=} \PY{n}{np}\PY{o}{.}\PY{n}{sum}\PY{p}{(}\PY{n}{similarities}\PY{p}{)}\PY{o}{\PYZhy{}}\PY{l+m+mi}{1}
             \PY{n}{product}\PY{o}{=}\PY{l+m+mi}{1}
             \PY{n}{wtd\PYZus{}sum} \PY{o}{=} \PY{l+m+mi}{0} 
             
             \PY{k}{for} \PY{n}{i} \PY{o+ow}{in} \PY{n+nb}{range}\PY{p}{(}\PY{l+m+mi}{0}\PY{p}{,} \PY{n+nb}{len}\PY{p}{(}\PY{n}{indices}\PY{o}{.}\PY{n}{flatten}\PY{p}{(}\PY{p}{)}\PY{p}{)}\PY{p}{)}\PY{p}{:}
                 \PY{k}{if} \PY{n}{indices}\PY{o}{.}\PY{n}{flatten}\PY{p}{(}\PY{p}{)}\PY{p}{[}\PY{n}{i}\PY{p}{]}\PY{o}{+}\PY{l+m+mi}{1} \PY{o}{==} \PY{n}{user\PYZus{}id}\PY{p}{:}
                     \PY{k}{continue}\PY{p}{;}
                 \PY{k}{else}\PY{p}{:} 
                     \PY{n}{ratings\PYZus{}diff} \PY{o}{=} \PY{n}{ratings}\PY{o}{.}\PY{n}{iloc}\PY{p}{[}\PY{n}{indices}\PY{o}{.}\PY{n}{flatten}\PY{p}{(}\PY{p}{)}\PY{p}{[}\PY{n}{i}\PY{p}{]}\PY{p}{,}\PY{n}{item\PYZus{}id}\PY{o}{\PYZhy{}}\PY{l+m+mi}{1}\PY{p}{]}\PY{o}{\PYZhy{}}\PY{n}{np}\PY{o}{.}\PY{n}{mean}\PY{p}{(}\PY{n}{ratings}\PY{o}{.}\PY{n}{iloc}\PY{p}{[}\PY{n}{indices}\PY{o}{.}\PY{n}{flatten}\PY{p}{(}\PY{p}{)}\PY{p}{[}\PY{n}{i}\PY{p}{]}\PY{p}{,}\PY{p}{:}\PY{p}{]}\PY{p}{)}
                     \PY{n}{product} \PY{o}{=} \PY{n}{ratings\PYZus{}diff} \PY{o}{*} \PY{p}{(}\PY{n}{similarities}\PY{p}{[}\PY{n}{i}\PY{p}{]}\PY{p}{)}
                     \PY{n}{wtd\PYZus{}sum} \PY{o}{=} \PY{n}{wtd\PYZus{}sum} \PY{o}{+} \PY{n}{product}
             
             \PY{n}{prediction} \PY{o}{=} \PY{n+nb}{int}\PY{p}{(}\PY{n+nb}{round}\PY{p}{(}\PY{n}{mean\PYZus{}rating} \PY{o}{+} \PY{p}{(}\PY{n}{wtd\PYZus{}sum}\PY{o}{/}\PY{n}{sum\PYZus{}wt}\PY{p}{)}\PY{p}{)}\PY{p}{)}
             \PY{n+nb}{print}\PY{p}{(}\PY{l+s+s1}{\PYZsq{}}\PY{l+s+se}{\PYZbs{}n}\PY{l+s+s1}{Predicted rating for user }\PY{l+s+si}{\PYZob{}0\PYZcb{}}\PY{l+s+s1}{ \PYZhy{}\PYZgt{} item }\PY{l+s+si}{\PYZob{}1\PYZcb{}}\PY{l+s+s1}{: }\PY{l+s+si}{\PYZob{}2\PYZcb{}}\PY{l+s+s1}{\PYZsq{}}\PY{o}{.}\PY{n}{format}\PY{p}{(}\PY{n}{user\PYZus{}id}\PY{p}{,}\PY{n}{item\PYZus{}id}\PY{p}{,}\PY{n}{prediction}\PY{p}{)}\PY{p}{)}
         
             \PY{k}{return} \PY{n}{prediction}
\end{Verbatim}


    \begin{Verbatim}[commandchars=\\\{\}]
{\color{incolor}In [{\color{incolor}4}]:} \PY{c+c1}{\PYZsh{}make necesarry imports}
        \PY{k+kn}{import} \PY{n+nn}{numpy} \PY{k}{as} \PY{n+nn}{np}
        \PY{k+kn}{import} \PY{n+nn}{pandas} \PY{k}{as} \PY{n+nn}{pd}
        \PY{k+kn}{import} \PY{n+nn}{matplotlib}\PY{n+nn}{.}\PY{n+nn}{pyplot} \PY{k}{as} \PY{n+nn}{plt}
        \PY{k+kn}{import} \PY{n+nn}{sklearn}\PY{n+nn}{.}\PY{n+nn}{metrics} \PY{k}{as} \PY{n+nn}{metrics}
        \PY{k+kn}{import} \PY{n+nn}{numpy} \PY{k}{as} \PY{n+nn}{np}
        \PY{k+kn}{from} \PY{n+nn}{sklearn}\PY{n+nn}{.}\PY{n+nn}{neighbors} \PY{k}{import} \PY{n}{NearestNeighbors}
        \PY{k+kn}{from} \PY{n+nn}{scipy}\PY{n+nn}{.}\PY{n+nn}{spatial}\PY{n+nn}{.}\PY{n+nn}{distance} \PY{k}{import} \PY{n}{correlation}\PY{p}{,} \PY{n}{cosine}
        \PY{k+kn}{from} \PY{n+nn}{sklearn}\PY{n+nn}{.}\PY{n+nn}{metrics} \PY{k}{import} \PY{n}{pairwise\PYZus{}distances}
        \PY{k+kn}{from} \PY{n+nn}{sklearn}\PY{n+nn}{.}\PY{n+nn}{metrics} \PY{k}{import} \PY{n}{mean\PYZus{}squared\PYZus{}error}
        \PY{k+kn}{import} \PY{n+nn}{sys}\PY{o}{,} \PY{n+nn}{os}
        \PY{k+kn}{from} \PY{n+nn}{IPython}\PY{n+nn}{.}\PY{n+nn}{display} \PY{k}{import} \PY{n}{Image}
\end{Verbatim}



    % Add a bibliography block to the postdoc
    
    
    
    \end{document}
